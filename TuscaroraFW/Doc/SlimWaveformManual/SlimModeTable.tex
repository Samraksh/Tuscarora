 \begin{longtable}{ | c | c | c |}

\caption{Attributes of SlimMode}\label{table:SlimModeTable}	\\
 
     \hline
  \textbf{Attribute} & \textbf{Short Description} & \textbf{Default Value} \\ \hline
 
     
    
 UniqueName & \parbox[][][t]{7cm}{\vspace{6pt}\raggedright A unique name for the mode\vspace{6pt}} & 
                   "UNINITIALIZEDMODE" \\ \hline 

 Bandwidth &  \parbox[][][t]{7cm}{\vspace{6pt}\raggedright The bandwidth(Hz) of the signal being modeled.\vspace{6pt}} & 
                   20000 \\ \hline 
                   
 Tpb & \parbox{7cm}{\vspace{6pt}\raggedright Time it takes to transmit one bit of information. Only needed for modes specifying payload.\vspace{6pt}} & 
                   1000ns \\ \hline 	
                   
Duration & \parbox{7cm}{\vspace{6pt}\raggedright Total time it takes to transmit the section. Only needed for fixed length sections of a packet. Calculated internally for payload.\vspace{6pt}} & 
              1000ns\\ \hline            
  
FecF & \parbox{7cm}{\vspace{6pt}\raggedright The fraction of bits that the FEC can correct. A double value.\vspace{6pt}} & 
                  0.0\\ \hline 	
                                   
ErrorModPtr & \parbox{7cm}{\vspace{6pt}\raggedright Pointer to the SlimModeError object implementing SINR vs. BER characteristics. The default is of type TuscaroraDsssDbpskModeError.\vspace{6pt}} & 
               NULL\\ \hline       
               



%m\_noiseFigure & \parbox{5cm}{\centering Loss  } & 
%              MicroSeconds(0)\\ \hline               

       							  		
	   
 \end{longtable}   
