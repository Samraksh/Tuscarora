% %set page layout
\usepackage[
  %paperwidth=6.5in,
  %paperheight=10in,
   top=0.75in,
   %left=2cm,
   %right=2cm,
 % bottom=25mm,
 % inner=10mm,
 % outer=80mm,
  %marginparsep=7mm,
  %marginparwidth=48mm,
]{geometry}

\usepackage[table,usenames,dvipsnames]{xcolor}
\usepackage{listings}
\lstset{ %
language=C++,                % choose the language of the code
basicstyle=\footnotesize,       % the size of the fonts that are used for the code
numbers=left,                   % where to put the line-numbers
numberstyle=\footnotesize,      % the size of the fonts that are used for the line-numbers
stepnumber=1,                   % the step between two line-numbers. If it is 1 each line will be numbered
numbersep=5pt,                  % how far the line-numbers are from the code
backgroundcolor=\color{gray!15},  % choose the background color. You must add \usepackage{color}
commentstyle=\color{PineGreen},
keywordstyle=\color{magenta},
stringstyle=\color{purple},
showspaces=false,               % show spaces adding particular underscores
showstringspaces=false,         % underline spaces within strings
showtabs=false,                 % show tabs within strings adding particular underscores
frame=single,           % adds a frame around the code
tabsize=2,          % sets default tabsize to 2 spaces
captionpos=b,           % sets the caption-position to bottom
breaklines=true,        % sets automatic line breaking
breakatwhitespace=false,    % sets if automatic breaks should only happen at whitespace
escapeinside={\%*}{*)}          % if you want to add a comment within your code
}

%\usepackage[table]{xcolor}
\usepackage[colorlinks=true,urlcolor=blue, citecolor=blue, linkcolor=blue]{hyperref}
\usepackage[english]{babel}
\usepackage{blindtext}
%\usepackage{newtxtext,newtxmath}
\usepackage{mathptmx}% Times Roman font
\usepackage{textcomp}
\usepackage[scaled=.90]{helvet}% Helvetica, served as a model for arial

% % %stuff needed to parse doxygen files
%\usepackage{import}	% the import package helsp to find Doxygen files in the latex subdirectory
%\usepackage{doxygen}
\usepackage[pdftex]{graphicx}   


\usepackage{cleveref}
%\usepackage[capitalize]{cleveref}
\crefname{figure}{Fig.}{Figs.}
\Crefname{figure}{Fig.}{Figs.}
%\crefname{algorithm}{Algorithm}{Algorithms}

%\renewcommand{\DoxyLabelFont}{%
%	\fontseries{bc}\selectfont%
%	\color{darkgray}%
%}
\newcommand{\+}{\discretionary{\mbox{\scriptsize$\hookleftarrow$}}{}{}}



% % % % % % % %lets define some cool colors
\definecolor{lightred}{rgb}{1.0,0.4,0.4}   % Light Redish Color



% % %Reformat the title fonts
\usepackage{titlesec}
%{\chaptertitlename\ \thechapter}{10pt}{\Huge}
\titleformat{\chapter}
  {\normalfont\sffamily\huge\bfseries\color{blue}}
  {\thechapter}{1em}{}
  
\titleformat{\section}
  {\normalfont\sffamily\Large\bfseries\color{cyan}}
  {\thesection}{1em}{}

\titleformat{\subsection}
{\normalfont\sffamily\large\bfseries\color{gray!125}}
{}{1em}{}
%{\textnormal{\roman{subsection}.}}{1em}{}


%\setkomafont{captionlabel}{


% % % % %Make header and footer

%\usepackage{fancyhdr}
%\usepackage[automark,headsepline,footsepline]{scrlayer-scrpage}
\usepackage{scrpage2}
%\automark{section}

\usepackage{parskip}

\clearscrheadfoot
\pagestyle{scrheadings}

%\lehead[Samraksh]{Samraksh}        % equal page, right position (inner) 
%\lohead[Samraksh]{Samraksh}       % odd   page, left  position (inner) 
%\lehead[]{this is page \pagemark} % equal page, left (outer) position
%\rohead[]{this is page \pagemark}
% definitions/configuration for the footer
\lefoot[\pagemark \ \ $\arrowvert$]{\pagemark \ \ $\arrowvert$}  % equal page, center position
\lofoot[\pagemark \ \ $\arrowvert$] {\pagemark \ \ $\arrowvert$}    
\refoot[\color{red} Limited Release, Do Not Redistribute.]{\color{red} Limited Release, Do Not Redistribute.}  % equal page, center position
\rofoot[\color{red} Limited Release, Do Not Redistribute.]{\color{red} Limited Release, Do Not Redistribute.}


\renewcommand*{\chapterpagestyle}{plain}
\addto{\captionsenglish}{\renewcommand{\bibname}{List of References}}

% % % % % %redefine some commands

%\renewcommand{\url}{ \color{lightred} {\url}}

\makeindex
\makeglossary

% To control the appereance of CPP variables within text
\newcommand{\CPPClassName}[1]{\emph{#1}}
\newcommand{\CPPVarName}[1]{\emph{#1}}
\newcommand{\CPPFuncName}[1]{\emph{#1}}
\newcommand{\CPPConstant}[1]{\emph{#1}}

\newcommand{\TuscConcept}[1]{``#1''}
\newcommand{\TuscDocuments}[1]{``#1''}

%\usepackage[obeyspaces,spaces]{url}
\usepackage{hyperref}
\newcommand{\FilePath}[1]{``\path{#1}''}
% to quickly comment text during revisions ;-)
% Select what to do with todonotes: 
% \usepackage[disable]{todonotes} % notes not showed
\usepackage[draft]{todonotes}   % notes showed
\usepackage{xstring}
%\newcommand{\bk}[1]{}
\newcommand{\bk}[1]{\todo[inline]{(BK: #1)}}
\newcommand{\bora}[2]{
	\IfEqCase{#1} {
		{1}{\todo[inline]{(BK: #2)}}
		{0}{}
	}[\PackageError{bora}{Undefined option to bora: #1}{}]%
}

\usepackage{booktabs}
\usepackage{multirow}
\usepackage{longtable}
