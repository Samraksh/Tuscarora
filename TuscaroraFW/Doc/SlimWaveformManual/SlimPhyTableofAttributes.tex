 \begin{longtable}{ | c | c | c |}

\caption{Attributes of TuscaroraSlimPhy}\label{table:SlimPhyTableofAttributes}	\\
 
     \hline
  \textbf{Attribute} & \textbf{Short Description} & \textbf{Default Value} \\ \hline
 
     
    
 EnergyDetectionThreshold & \parbox[][][t]{5cm}{\vspace{6pt}\raggedright The energy of a received signal should be higher than 
                   this threshold (dbm) to allow the PHY layer to detect the signal.} & 
                   -96.0 \\ \hline 

 TxGain & \parbox[][][t]{5cm}{\vspace{6pt}\raggedright Transmission gain (dB).} & 
                   1.0 \\ \hline 

 RxGain & \parbox[][][t]{5cm}{\vspace{6pt}\raggedright Reception gain (dB).} & 
                   1.0 \\ \hline 

 TxPowerLevels & \parbox[][][t]{5cm}{\vspace{6pt}\raggedright Number of transmission power levels available between 
                   TxPowerStart and TxPowerEnd included.} & 
                   1 \\ \hline 

 TxPowerEnd & \parbox[][][t]{5cm}{\vspace{6pt}\raggedright Maximum available transmission level (dbm).} & 
                   16.0206 \\ \hline 

 TxPowerStart & \parbox[][][t]{5cm}{\vspace{6pt}\raggedright Minimum available transmission level (dbm).} & 
                   16.0206 \\ \hline 

 RxNoiseFigure & \parbox[][][t]{5cm}{\vspace{6pt}\raggedright Loss (dB) in the signal to noise ratio due to non-idealities in the receiver.} & 
                   0 \\ \hline 

 StateHelperPtr & \parbox[][][t]{5cm}{\vspace{6pt}\raggedright The pointer to the helper class object that keeps the state of the phy layer} & 
                   PointerValue () \\ \hline 

 ChannelSwitchDelay & \parbox[][][t]{5cm}{\vspace{6pt}\raggedright Time it takes to switch channels.} & 
                   250000ns \\ \hline 

 ChannelNumber & \parbox[][][t]{5cm}{\vspace{6pt}\raggedright Channel number among multiple orthogonal channels.}
 %Channel center frequency = Channel starting frequency + 5 MHz * nch} & 
                  & 1 \\ \hline 

 Frequency & \parbox[][][t]{5cm}{\vspace{6pt}\raggedright The central operating frequency.} & 
                   2407 \\ \hline 

 AntennaPtr & \parbox[][][t]{5cm}{\vspace{6pt}\raggedright The pointer pointing to the associated antenna object.} & 
                  IsotropicAntennaModel \\ \hline 

% Transmitters & \parbox[][][t]{5cm}{\vspace{6pt}\raggedright The number of transmitters.} & 
%                   UintegerValue (1) \\ \hline 
%
% Receivers & \parbox[][][t]{5cm}{\vspace{6pt}\raggedright The number of receivers.} & 
%                   UintegerValue (1) \\ \hline 
%
% ShortGuardEnabled & \parbox[][][t]{5cm}{\vspace{6pt}\raggedright Whether or not short guard interval is enabled.} & 
%                   BooleanValue (false) \\ \hline 
%
% LdpcEnabled & \parbox[][][t]{5cm}{\vspace{6pt}\raggedright Whether or not LDPC is enabled.} & 
%                   BooleanValue (false) \\ \hline 
%
% STBCEnabled & \parbox[][][t]{5cm}{\vspace{6pt}\raggedright Whether or not STBC is enabled.} & 
%                   BooleanValue (false) \\ \hline 
%
% GreenfieldEnabled & \parbox[][][t]{5cm}{\vspace{6pt}\raggedright Whether or not STBC is enabled.} & 
%                   BooleanValue (false) \\ \hline 
%
% ChannelBonding & \parbox[][][t]{5cm}{\vspace{6pt}\raggedright Whether 20MHz or 40MHz.} & 
%                   BooleanValue (false) \\ \hline 
%
% CenterFrequency & \parbox[][][t]{5cm}{\vspace{6pt}\raggedright The center frequency of the channel} & 
%				   DoubleValue (2407) \\ \hline 
%
% RateDefinitions & \parbox[][][t]{5cm}{\vspace{6pt}\raggedright The objects that hold rates supported} & 
%				   PointerValue () \\ \hline 

 
				   

			   
				   
				   
	   
 \end{longtable}   
