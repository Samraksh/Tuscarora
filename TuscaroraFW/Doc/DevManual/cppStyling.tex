
\usepackage{listings}

\lstdefinestyle{boralargefile}{
	language=C++,                % choose the language of the code
	basicstyle=\footnotesize,       % the size of the fonts that are used for the code
	numbers=left,                   % where to put the line-numbers
	numberstyle=\footnotesize,      % the size of the fonts that are used for the line-numbers
	stepnumber=1,                   % the step between two line-numbers. If it is 1 each line will be numbered
	numbersep=5pt,                  % how far the line-numbers are from the code
	backgroundcolor=\color{gray!15},  % choose the background color. You must add \usepackage{color}
	commentstyle=\color{PineGreen},
	keywordstyle=\color{magenta},
	stringstyle=\color{purple},
	showspaces=false,               % show spaces adding particular underscores
	showstringspaces=false,         % underline spaces within strings
	showtabs=false,                 % show tabs within strings adding particular underscores
	frame=single,           % adds a frame around the code
	tabsize=2,          % sets default tabsize to 2 spaces
	captionpos=b,           % sets the caption-position to bottom
	breaklines=true,        % sets automatic line breaking
	breakatwhitespace=false,    % sets if automatic breaks should only happen at whitespace
	escapeinside={\%*}{*)}          % if you want to add a comment within your code
}

\lstdefinestyle{boralargefileNoNumbers}{
	language=C++,                % choose the language of the code
	basicstyle=\footnotesize,       % the size of the fonts that are used for the code
	backgroundcolor=\color{gray!15},  % choose the background color. You must add \usepackage{color}
	commentstyle=\color{PineGreen},
	keywordstyle=\color{magenta},
	stringstyle=\color{purple},
	showspaces=false,               % show spaces adding particular underscores
	showstringspaces=false,         % underline spaces within strings
	showtabs=false,                 % show tabs within strings adding particular underscores
	frame=single,           % adds a frame around the code
	tabsize=2,          % sets default tabsize to 2 spaces
	captionpos=b,           % sets the caption-position to bottom
	breaklines=true,        % sets automatic line breaking
	breakatwhitespace=false,    % sets if automatic breaks should only happen at whitespace
	escapeinside={\%*}{*)}          % if you want to add a comment within your code
}


\lstdefinestyle{tt}{
	language=C++,                % choose the language of the code
	basicstyle=\footnotesize,       % the size of the fonts that are used for the code
	backgroundcolor=\color{gray!15},  % choose the background color. You must add \usepackage{color}
	commentstyle=\color{PineGreen},
	keywordstyle=\color{magenta},
	stringstyle=\color{purple},
	showspaces=false,               % show spaces adding particular underscores
	showstringspaces=false,         % underline spaces within strings
	showtabs=false,                 % show tabs within strings adding particular underscores
	frame=single,           % adds a frame around the code
	tabsize=2,          % sets default tabsize to 2 spaces
	captionpos=b,           % sets the caption-position to bottom
	breaklines=true,        % sets automatic line breaking
	breakatwhitespace=false,    % sets if automatic breaks should only happen at whitespace
	escapeinside={\%*}{*)}          % if you want to add a comment within your code
	}

\lstdefinestyle{rm}{basicstyle=\ttfamily,keywordstyle=\slshape,language=[LaTeX]{TeX}}

\newcommand{\CPPSingleLine}[1]{\lstinline[style=tt]{#1}}

\definecolor{ColorOfClassNames}{RGB}{0, 80, 50}
\definecolor{ColorOfTemplateNames}{RGB}{100, 70, 50}

% To control the appereance of CPP variables within text
\newcommand{\CPPClassName}[1]{{\color{ColorOfClassNames}\texttt{#1}}}
\newcommand{\CPPVarName}[1]{{\color{blue}\emph{#1}}}
\newcommand{\CPPFuncName}[1]{{\textbf{\emph{#1}}}}
\newcommand{\CPPConstant}[1]{\emph{#1}}
\newcommand{\CPPTemplateName}[1]{{\color{ColorOfTemplateNames}\texttt{#1}}}

\newcommand{\CPPKeyword}[1]{{\color{red}\emph{#1}}}




